\documentclass[conference,a4paper,10pt]{IEEEtran}
\IEEEoverridecommandlockouts
% The preceding line is only needed to identify funding in the first footnote. If that is unneeded, please comment it out.
\usepackage{cite}
\usepackage{amsmath,amssymb,amsfonts}
\usepackage{algorithmic}
\usepackage{graphicx}
\usepackage{textcomp}
\usepackage{xcolor}
\usepackage{url}
\usepackage{multirow}
\def\BibTeX{{\rm B\kern-.05em{\sc i\kern-.025em b}\kern-.08em
    T\kern-.1667em\lower.7ex\hbox{E}\kern-.125emX}}
\begin{document}

\title{SafeDE: a flexible \underline{D}iversity \underline{E}nforcement hardware module for light-lockstepping}


\author{
  \IEEEauthorblockN{Francisco Bas$^{\dagger,\ddagger}$, Sergi Alcaide$^{\dagger}$, Ruben Lorenzo$^{\dagger}$, Guillem Cabo$^{\dagger}$, Guillermo Gil$^{\dagger}$,\\ 
                    Oriol Sala$^{\dagger,\ddagger}$, Fabio Mazzocchetti$^{\dagger}$, David Trilla$^{\dagger}$, Jaume Abella$^{\dagger}$}
  \IEEEauthorblockA{
    $^\dagger$ Barcelona Supercomputing Center (BSC). Barcelona, Spain\\
    $^\ddagger$ Universitat Polit\`{e}cnica de Catalunya (UPC), Barcelona, Spain
\vspace{-0.1cm}
}}

\maketitle

\begin{abstract}
Safety-related systems, such as those in automotive, avionics and space, impose the existence of appropriate safety measures to meet the safety requirements of the system. In the case of the highest integrity level functionalities (e.g. ASIL-D in automotive), diverse redundancy must be deployed to avoid unreasonable risk of a single fault leading the system to a failure (e.g. using lockstepped cores). 
%This is achieved including features such as error correction codes for storage, and lockstep execution for computing units (e.g. cores). However, in the case of lockstep, 
However, existing lockstep solutions are either (1) highly intrusive and inflexible coupling two cores with hardware means, or (2) costly in terms of execution time and monitoring if a software monitor thread checks that cores running redundantly preserve sufficient staggering.

This paper presents \textbf{SafeDE}, a \underline{D}iversity \underline{E}nforcement hardware module providing light-lockstep support by means of a non-intrusive and flexible hardware module that preserves staggering across cores running redundant threads, thus bringing time diversity. SafeDE reconciles the lightness and flexibility of software-only solutions, even allowing using the cores without any lockstepping, as well as the tighter staggering of hardware-only solutions that allow using staggering values of few cycles, instead of hundreds of microseconds, as for software-only solutions.
Our integration of SafeDE in a RISC-V FPGA-based space multicore from Cobham Gaisler shows that staggering is effectively preserved, and SafeDE overheads are negligible in terms of area and performance due to staggering.
\end{abstract}

%\begin{IEEEkeywords}
%functional safety, redundancy, diversity, faults
%\end{IEEEkeywords}

%\linespread{0.99}
%\selectfont

\section{Introduction}

Increased autonomy levels and improved features in cars, satellites and planes lead to increasing performance demands for those systems. Existing multicores and accelerators deliver the level of performance needed. However, safety-related systems such as those in automotive, space and avionics need to meet specific safety requirements for their adoption. Those requirements are dictated by the specific safety integrity level of the functionality at hand, and are particularly demanding for the highest levels (e.g. Automotive Safety Integrity Level, ASIL, D in automotive~\cite{ISO26262}). 

A key safety requirement for the highest integrity level systems is the capability of not causing a failure due to a single fault. Those faults are normally mitigated by means of diverse redundancy~\cite{ClassicDiversity}, so that redundancy exists but it is sufficiently diverse so that a single fault affecting all replicas would cause different effects, thus allowing, at least, detecting the fault. 
%Such fault detection should allow recovering or reaching a safe state before the fault leads to a failure. 

Solutions for diverse redundancy often relate to Error Correction Codes (ECC) for storage and Cyclic Redundancy Coding (CRC) for communications~\cite{SergiIOLTS}. Computing elements (e.g. cores), instead, usually require full replication and thus, resort to dual (DMR)~\cite{Mukherjee2002,Gomaa2003,dynamic_coupled_cores} or triple modular redundancy (TMR)~\cite{Iturbe2019}. However, such redundancy is not enough and diversity is also needed to avoid that a single fault (e.g. a voltage droop or a permanent fault) affects redundant instances identically. Such diverse redundancy is achieved with lockstepping, where two (or more) identical cores execute the same software redundantly, but with some staggering among them, so that the state of the cores differs at any point in time, and thus, a fault cannot produce the same error in all redundant copies, which could go unnoticed otherwise.

Lockstepping, in the form of Dual Core LockStepping (DCLS)~\cite{infineon_aurix,STlockstep,RendundancyASILD}, is generally implemented at hardware level tying two cores together operating with few cycles of staggering (i.e. the head core executes the same software $N$ cycles ahead of the trail core). External requests are generated by one of the cores (e.g. head one), but not sent until compared to those of the other core (e.g. trail one). Upon a match, loads, stores, interrupts and any other type of request is sent, but just once (not redundantly). Responses are duplicated and delivered to both cores preserving the staggering (i.e. delivering them $N$ cycles later to the trail core). This ensures consistent states across lockstepped cores, but with some staggering to preserve diversity. Such a solution, however, makes only one of the cores be visible at software level, and precludes the user from using those cores independently.
%implies using modified cores w.r.t. those used without lockstepping, and precludes the ability to use those cores to run different software not needing lockstepping.

Light-weight software-only lockstepping has been proposed to reduce the cost of full lockstepping~\cite{SergiDFT}. Such light-weight lockstepping resorts to software redundancy, and to the existence of a software monitor enforcing staggering across redundant threads. While such a solution has been proven effective, and compatible with Commercial Off-The-Shelf (COTS) processors, it requires native lockstepping (hardware-based) for the core running the monitor, and imposes some non-negligible staggering (e.g. 100$\mu$s) to allow the monitor to collect information of the progress of redundant threads without causing too high relative interference.

Overall, both hardware-based and light software-based lockstepping pose a number of limitations to achieve diverse redundancy. This paper, addresses this challenge by proposing a different tradeoff achieving most of the benefits of both approaches with low cost.

This paper presents \emph{SafeDE}\footnote{Available as an open-source component in https://bsccaos.github.io~\cite{SafeTIWebsite}.}, a \underline{D}iversity \underline{E}nforcement hardware module providing light-lockstep support by means of a non-intrusive and flexible (programmable) hardware module that preserves staggering across cores running redundant threads, thus bringing time diversity. In particular, SafeDE is a tiny hardware module performing the same monitoring tasks as the software-only solution, but with a much lower staggering (just few cycles instead of 100$\mu$s), and without requiring native lockstepped cores. 
%Of course, while the software-only solution can be applied on COTS, SafeDE needs to be integrated in the SoC.
Compared to native hardware lockstepping, SafeDE can be integrated without modifying IP cores, thus with limited intrusiveness, and allows using cores independently instead of always in lockstep mode.
In particular, the contributions of this paper are as follows:
\begin{itemize}
\item We present SafeDE, a new hardware/software scheme for efficient, flexible and lowly-intrusive light lockstepping to achieve diverse redundancy.
\item We implement and verify SafeDE in VHDL.
\item We successfully integrate SafeDE in a space SoC based on Cobham Gaisler's NOEL-V cores, implemented in a FPGA, which is already a commercial setup for this platform reaching commercial readiness by early 2022~\cite{DeRISCjournal}.
\end{itemize}

The rest of the paper is organized as follows. Section~\ref{sec:back} provides some background. SafeDE is presented in Section~\ref{sec:dimmo} and evaluated in Section~\ref{sec:eval}. Section~\ref{sec:rel} reviews related work. Section~\ref{sec:concl} concludes this paper.


\section{Background}
\label{sec:back}

This section reviews the main concepts and approaches related to lockstep execution relevant for this work.

\subsection{Redundancy, Diversity and Sphere of Replication}
The design, as well as the verification and validation (V\&V) stages for safety-related systems arguably remove unreasonable risk of any kind of systematic fault, either software or hardware related. However, random hardware faults are unavoidable in nature, so they must be managed with suitable safety measures, being diverse redundancy a mandatory safety measure for the highest Safety Integrity Levels (SIL for short).

Diverse redundancy can be realized by using diverse hardware and/or software, but these approaches may double part of the design, and V\&V costs. Alternatively, diverse redundancy can also be realized by executing the same software in identical hardware, but with some staggering, hence guaranteeing that replicas hold different state at any time instant so that any common fault will lead to diverse errors (if any). This approach, if realized with two cores, is referred to as Dual Core LockStepping (DCLS) and is used in several commercial processors~\cite{infineon_aurix,STlockstep,RendundancyASILD}.

The sphere of replication determines what outputs of the replicas are compared to detect errors. In the case of hardware-based lockstep, such sphere includes only a core, so any off-core activity (data fetch or store beyond in-core caches, I/O activity, interrupts, etc.) is compared across cores for error detection. In the case of light-weight lockstep execution, it is limited to programs or code regions without I/O, and detects errors by comparing data outputs at the end of the execution.


\subsection{Lockstep Schemes}

\begin{figure}[t!]
\centering
\begin{tabular}{cc}
  \includegraphics[width=0.37\columnwidth]{imgs/HWlockstep.png} & 
  \includegraphics[width=0.53\columnwidth]{imgs/SWlockstep.png} \\
  (a) Hardware-only & (b) Software-only \\
\end{tabular}
  \caption{Schematic of the existing lockstep schemes.}
  \label{fig:HWSWlockstep}
\end{figure}

\textbf{Tight hardware-based lockstep execution}. This approach, implemented in processors such as, for instance, the Infineon AURIX family~\cite{infineon_aurix}, uses two physical cores (head and trail cores) out of which only one (e.g., head core) is visible at user level and the other one (e.g., trail core) is a shadow core. They perform \emph{exactly} the same activity but shifted by $N$ cycles, 
so that the state of the trail core matches that of the head core $N$ cycles before.
External requests (data load/store, interrupts, etc.) are compared before being exposed externally, hence needing some buffering to store head requests during $N$ cycles. Analogously, responses are delivered immediately to the head core when they arrive, but queued during $N$ cycles before being delivered to the trail core, which also needs some buffering. Such scheme is depicted in Figure~\ref{fig:HWSWlockstep}(a). Note that staggering is typically low (e.g., 2 or 3 cycles) to keep buffering overheads low.

\textbf{Light-weight software-based lockstep execution}. In this approach, redundancy is created at software level by running a given program (task) twice on different cores~\cite{SergiDFT}. Those task replicas run along with a monitor thread, which is deployed in a third core, to enforce some staggering across replicas so that one of them becomes the head and the other one the trail task (and core). Note that the monitor itself is unprotected, so this approach requires the execution of the monitor to occur on a core with hardware-based lockstepping, either in the same or another chip. 
In detail, the operation of this scheme is as follows (see Figure~\ref{fig:HWSWlockstep}(b)): the monitor schedules redundant processes (replicas) in two different cores, but only allows the head core to make progress. The monitor collects the number of instructions executed by each core ($\#instr$ in the figure), and only allows the trail core to execute its task if $\#instr_{head} - \#instr_{trail}$, namely the staggering in terms of number of instructions, exceeds a given threshold $TH_{stag}$. Such condition is checked by the monitor every $T_{check}$ cycles to decide whether the trail core is allowed to proceed during the following interval. Only when the head core finishes its execution, the trail core is allowed to run unrestrictedly. Results from both executions are compared when both cores complete their execution. 
Note that between two consecutive checks of the staggering, the trail core could execute up to $T_{check} \cdot CommitWidth$ instructions, where $CommitWidth$ stands for the maximum number of instructions that can be retired per cycle. $TH_{stag}$ must be strictly higher than $T_{check} \cdot CommitWidth$.
As shown in \cite{SergiDFT}, due to the software overheads to collect $\#instr_{head}$ and $\#instr_{trail}$, and to stall a process -- if needed, $TH_{stag}$ must be a number of instructions taking at least around 100$\mu$s to run.

\section{SafeDE: a \underline{D}iversity \underline{E}nforcement hardware module}
\label{sec:dimmo}

This section presents the architecture of SafeDE, its features and limitations, its extension towards N-modular redundancy, and its implementation and integration (both hardware and software) details.

\subsection{SafeDE Architecture}
\label{sec:arch}

\begin{figure}[t!]
\centering
  \includegraphics[width=1\columnwidth]{imgs/dimmo.png} 
  \caption{SafeDE architecture.}
  \label{fig:dimmo}
\end{figure}

SafeDE builds on the light-weight lockstepping concept, which has only been implemented in software so far~\cite{SergiDFT}, with the aim of preserving its advantages and removing its main limitations, i.e., the need for an extra core to run the monitor and a long feedback loop imposing a large staggering. SafeDE is architected as a tiny component connected to the two monitored cores, as shown in Figure~\ref{fig:dimmo}. SafeDE collects the instruction counts from the two cores, namely $\#instr_{head}$ and $\#instr_{trail}$, and generates the stall signal for the trail core. As for the software-only solution, SafeDE checks whether the head core is at least $TH_{stag}$ instructions ahead of the trail one. If this is not the case, the stall signal for the trail core is raised, which stalls its pipeline by stalling one or several of its stages (e.g., stalling the commit stage).

\textbf{SafeDE parameters}. The configuration registers of SafeDE are $TH_{stag}$, $active$, $CritSec1$ and $CritSec2$, and they operate as follows:
\begin{itemize}
\item $TH_{stag}$ corresponds to the minimum staggering (in terms of number of instructions) to be enforced between the head and trail cores. Typically, it has a very low value (e.g., 10 instructions), hence implementing a staggering distance much smaller than that of software-only solutions, and comparable to that of tight hardware-based lockstep execution.
\item $active$ determines whether SafeDE is active. If this signal is reset, SafeDE is completely neutral since it can never stall that trail core.
\item $CritSec1$ ($CritSec2$) is set by the head (trail) core when it enters the code region needing lockstep, and reset when leaving it. Hence, lockstepping must be enforced when $CritSec1$ and $CritSec2$ are both set, as this indicates that both cores are executing the code region needing lockstep execution.
\end{itemize}

\textbf{SafeDE operation}. While $active=0$, SafeDE is inactive. Eventually, $TH_{stag}$ is programmed and $active$ is set, hence activating SafeDE. Activating SafeDE automatically resets $CritSec1$ and $CritSec2$ keeping SafeDE ready but innocuous until $CritSec2$ is activated. Eventually, one of the two cores activates its $CritSec$ register, becomes the head core, and its instruction counter ($\#instr_{head}$) is reset and starts counting. Whenever the other core sets its $CritSec$ register, it becomes the trail core, and its instruction counter ($\#instr_{trail}$) is also reset. If the head core is not ahead $TH_{stag}$ instructions of the trail core, SafeDE sets the stall signal for the trail core. Note that, since any of the two cores could be the trail core depending on which one sets its $CritSec$ first, the $stall$ signal exists for both cores. Note that, if the staggering is too low when the trail core sets its $CritSec$ ($\#instr_{head} - \#instr_{trail} < TH_{stag}$), the $stall$ signal for the trail core will be raised immediately. Whenever the staggering is enough, the trail core is allowed to resume its execution.
Note that SafeDE checks every cycle whether the staggering is enough. This allows using tiny staggering ($TH_{stag}$) values, in contrast with the large staggering needed by the software-only solution. Moreover, SafeDE controls this condition, hence not needing any additional core to run any monitor software.
Also note that by performing such check every cycle, negligible switching power is induced since $\#instr_{head}$ and $\#instr_{trail}$ barely change.
Eventually, the head core reaches the end of its protected code region and resets its $CritSec$ register. At that point, SafeDE becomes innocuous again not raising any stall signal, hence letting the trail core reaching also the end of its protected region.

\textbf{Software process}. At software level, end users need to configure and set SafeDE active with the corresponding driver, and typically, use an API to schedule both redundant processes to the corresponding cores managing $CritSec$ registers accordingly. Those software components are described later in this section in the context of bare metal and Linux integrations.


\subsection{Features and Limitations Analysis}

This section presents the key features and limitations of SafeDE, and how they compare against the software-only solution~\cite{SergiDFT}. 

\subsubsection{SafeDE features}
\begin{itemize}
\item \textbf{Low cost}. SafeDE is a tiny hardware module implementing light-weight lockstep execution that avoids the need for an extra core to run a monitor at specific (tight) time intervals, as opposed to the software-only solution.
\item \textbf{Low staggering}. By controlling the feedback loop at hardware level, it is checked every cycle and hence, staggering can be kept to a minimum (e.g., 10 or 20 instructions). Note that the software-only solution needs a staggering value of many thousands of instructions to reach a staggering above 100$\mu$s, as discussed before.
\item \textbf{Flexibility}. SafeDE can be easily enabled and disabled. Hence, the main overheads relate to the creation of the redundant processes at software level, as needed in the context of light-weight lockstep execution, but not to the actual implementation of SafeDE, which does not impose further limitations.
\item \textbf{Low intrusiveness}. SafeDE needs some signals to be exported from cores, such as those to read and reset instruction counts, and the pipeline stalling signal. These modifications are much lighter than those needed in the case of tight lockstep execution. While the software-only solution does not require any hardware change, as opposed to SafeDE, it may need modifying the operating system to enable the management of the instruction counts remotely from other cores. SafeDE does not need any such operating system modification.
\end{itemize}

\subsubsection{SafeDE limitations}
\begin{itemize}
\item \textbf{Non-null intrusiveness}. While hardware modifications needed by SafeDE are light, it needs hardware support, and hence, cannot be used on COTS multicores, as opposed to the software-only solution.
\item \textbf{Limited applicability}. Light-weight lockstepping relies on the redundant processes executing identical instruction streams to guarantee the effectiveness of the approach. While this is usually the case, it precludes the use of this scheme for programs whose control path is non-deterministic (e.g., based on random choices independent across redundant processes). 
Also, SafeDE may not be used for parallel programs if the number of instructions of any thread may vary depending on the order in which they get a specific lock, since this could make redundant threads execute a different number of instructions. 
Also, since light-weight lockstepping exposes all activity redundantly, it should not be used along with I/O operations that may change the functionality of the system if repeated. In any case, note that those limitations relate to the light-weight lockstepping scheme rather than to SafeDE, and hence, affect also the software-only solution.
\item \textbf{Limited diversity}. By using two cores with staggered execution, SafeDE, as well as the software-only solution, provide physical and time diversity. However, if CCFs can be induced by the core design (e.g., physically weak gates identical in both cores), other types of diversity, such as layout diversity, are needed, and such diversity cannot be reached without appropriate (and intrusive) hardware modifications.
\item \textbf{SafeDE hardening}. SafeDE must be hardened to mitigate the risk of a single fault in SafeDE leading to a failure. Alternatively, SafeDE can also be implemented with physical diverse redundancy, as for tight lockstepping replicating the scheme in Figure~\ref{fig:HWSWlockstep}(a), but applied to SafeDE instead of to the cores.
\end{itemize}

\subsubsection{Scope of applicability}
As discussed before, light-weight lockstepping has limited applicability, hence affecting both SafeDE and the software-only solution. Therefore, SafeDE can only be used for some code regions rather than for the full program. Code regions exercising light-weight lockstepping limitations must be run on cores implementing tight lockstepping. Hence, at least two cores must implement tight hardware-based lockstepping. However, compute intensive parts of the code can be managed with SafeDE. This opens the door to using deployments with two tightly lockstepped cores (e.g., to run I/O-related code regions) and the remaining cores building on SafeDE. For instance, an 8-core multicore would offer 7 user-visible cores by shadowing only one of the cores for tight lockstepping. Hence, one could run a varying number of lockstepped and non-lockstepped tasks, ranging from 4 lockstepped to 1 lockstepped and 6 non-lockstepped. Instead, if tight lockstepping is used for all cores, then only 4 cores are visible at user level regardless of whether tasks need such support.

\subsection{Towards N-modular Redundancy}

\begin{figure}[t!]
\centering
  \includegraphics[width=1\columnwidth]{imgs/Nmodular.png} 
  \caption{N-modular redundancy scheme example with SafeDE.}
  \label{fig:Nmodular}
\end{figure}


Note that, while we implement and assess SafeDE in the context of dual-modular redundancy (DMR), it can be easily extended to N-modular redundancy, which may be needed for $N>2$ in some domains such as, for instance, avionics or medical, where 3-modular or even 5-modular redundancy may be needed. 

Regardless of the value of $N$, SafeDE must manage the $N$ cores so that one of them is the head core, one of them is the trail core, and the remaining ones inherit both, head and trail behavior. For instance, in the case of triple-modular redundancy (TMR), core 1 is the head core w.r.t. core 2, core 2 is the trail core w.r.t. core 1 and the head core w.r.t. core 3, and core 3 is the trail core w.r.t. core 2.
Therefore, given $N$ cores, SafeDE must activate the $stall_i$ signal for $core_i$ if the following condition holds: $\#instr_{i-1} - \#instr_{i} < TH_{stag}$, where $\#instr_{i-1}$ and $\#instr_i$ correspond to the instruction counts of $core_{i-1}$ and $core_i$ respectively, and $1 \le i < N$.

In the context of N-modular redundancy, note that the operation with $CritSec$ is analogous to DMR, hence cores take the role of $core_i$ when they are the $i^{th}$ core setting their corresponding $CritSec$ register. A core $core_i$ cannot be further stalled when $core_{i-1}$ leaves its critical region by resetting its $CritSec$ register. However, the remaining cores (from $core_{i+1}$ to $core_{N}$) can still be stalled if their staggering becomes too low w.r.t. their respective head cores.

One potential realization of the overall concept with flexible N-modular redundancy could impose that the cores in the multicore are physically paired for SafeDE operation so that $core_1$ is the head of $core_2$, $core_2$ of $core_3$, and so on and so forth. Then, $N-1$ SafeDE modules are deployed connecting each pair of consecutive cores. Finally, by activating appropriate SafeDE modules, one could have any combination of N-modular redundancy at will. For instance, Figure~\ref{fig:Nmodular} illustrates a case where an 8-core multicore uses 3 cores with TMR (cores 1-3), two pairs of 2 cores with DMR (cores 4-5 and 7-8), and 1 core running independently (core 6). This is achieved by enabling specific SafeDE modules (light-colored ones) and keeping others inactive (black ones).



\subsection{Implementation and Integration}
\label{sec:integ}

\begin{figure}[t!]
\centering
  \includegraphics[width=1\columnwidth]{imgs/system.png} 
  \caption{High-level representation of SafeDE integrated into the system.}
  \label{fig:system}
\end{figure}

As a proof of concept, we have integrated SafeDE in an industrial space MultiProcessor System on Chip (MPSoC) based on CAES Gaisler RISC-V NOEL-V cores~\cite{SELENEgit}. This platform consists of a consistent set of reusable VHDL IP cores by Gaisler whose main interface is a set of common on-chip buses. Those buses implement the standard AMBA 2.0, and SafeDE has been implemented in VHDL as another IP core compatible with such bus interface.

\subsubsection{System on Chip}
SafeDE is integrated and evaluated in a specific MPSoC instance including 2 Gaisler's NOEL-V 64-bit cores. Those cores are dual-issue, implement the RISC-V Instruction Set Architecture (ISA), include 7 pipeline stages, and local L1 data and instruction caches. Cores are connected among them, and to a shared L2 cache and the memory subsystem through a 128-bit AMBA Advanced High-performance Bus (AHB). Components requiring low bandwidth, such as for instance, SafeDE, are connected instead through an AMBA Advanced Peripheral Bus (APB).

\subsubsection{Hardware integration}
SafeDE interface builds on the standard APB interface to make it highly portable. In particular, SafeDE is an APB slave in the SoC.
SafeDE is also directly connected to the cores to collect their instruction counters, which are mapped to SafeDE as inputs, to stall the trail core whenever needed. The instruction counters determine the number of instructions executed by each core and used to compute the real staggering across head and trail cores. Regarding the stall signal produced by SafeDE, it is ORed with an internal core signal in charge of freezing the pipeline by not allowing register values to be updated. Overall, the only modifications needed in the cores include exporting the instruction counter\footnote{Note that the instruction counter, along with the cycle counter, are the main counters in any processor and are generally implemented in any SoC.} to SafeDE, and placing an OR gate appropriately to stall the pipeline whenever needed.
The SoC including SafeDE is depicted in Figure \ref{fig:system} for completeness.

\subsubsection{Configuration and operation}
SafeDE configuration registers, namely $TH_{stag}$, $active$, $CritSec1$ and $CritSec2$, are mapped to specific memory addresses. Their operation is detailed in Section~\ref{sec:arch}.
Apart from those functional registers, SafeDE also includes several statistics registers collecting information such as the maximum and minimum staggering observed, number of stall signal activations (i.e., how many times the $stall$ signal is raised), number of stall cycles for the trail core, executed instructions of each core, etc. 
Since SafeDE has an APB interface and memory-mapped registers, all its registers can be read and written with regular load and store operations.

\subsection{Software Integration}

We have developed the software interface to control the SafeDE IP. We considered two scenarios where SafeDE can be deployed: bare-metal systems, and systems with an operating system (Linux in our case). Hence, we implemented two software integrations. A C library, which is enough for a bare metal setup, and a driver for a Linux setup.

\subsubsection{Bare Metal Integration}
\label{sec:bare_metal}

To integrate SafeDE software in a bare metal setup we have created an API that consists of a C library to configure the internal SafeDE registers. The API contains functions to enable and reset SafeDE, configure the staggering, indicate when the critical section starts and finishes for each core, and gather the execution results (statistics).
At run time, SafeDE must be initialized first.
Such initialization includes configuring the staggering, and setting the reset and enable bits using the API. Whenever a core starts or finishes the critical region, a function from the API has to be executed to notify the SafeDE module. Once the task has finished, the statistics from the safe execution can be retrieved, again, using a function provided in the API library.

\subsubsection{Linux Integration}

In Linux, it is not possible to access directly the memory positions mapped to SafeDE registers from the user space. To allow the user to access SafeDE registers, we need to access the kernel space using a driver. We have programmed a Linux driver and a C library (API) that allow the user to communicate with the SafeDE module. 

For the Linux API, we use a set of functions analogous to those for the bare metal integration, but this time managed through the driver. The API calls communicate with the driver by writing the commands into a Linux device file, a special file in the Linux file system created during the driver initialization process. Later, the driver reads the commands from the device file and modifies the SafeDE registers accordingly. 

As stated before, one of the SafeDE limitations is that both cores have to execute the exact same instructions. However, in a preemptive operating system, a process in the critical region may be preempted. Since the critical region is active, instructions executed in that core would be counted as part of the critical region, hence de-synchronizing staggering in an arbitrary manner. Deactivating the corresponding $CritSec$ register would not be a better solution since it would not occur immediately, hence altering the instruction count anyway, and would lead to the virtual finalization of the lockstep execution. 
Some guidelines to avoid this situation are as follows:
\begin{enumerate}
    \item In Linux, each process has a mask indicating in which subset of the cores that process can be executed (a.k.a. process affinity). The particular case in which this subset is only one core, is named binding. In our strategy, we bind the two critical processes into two cores and modify all the other processes (including kernel processes) affinity to avoid preemption of the critical processes.
    \item In order to give the highest priority to the two redundant critical tasks so that they start immediately and unnecessary stalls do not occur, we must enqueue those tasks into the highest priority queue, which is SCHED\_FIFO queue. We perform that employing the Linux system call \textit{sched\_setscheduler()}. 
\end{enumerate}


Note that, if a real-time operating system (RTOS) for safety-related systems was used instead of Linux (e.g., fentISS' XtratuM, SYSGO's PikeOS, RTEMS), such problem would not exist and simpler mechanisms related to task scheduling and priority assignment would suffice to guarantee the non-preemptive execution of light-weight lockstepped processes.



\section{Evaluation}
\label{sec:eval}

We evaluate SafeDE by synthesizing the RISC-V multicore SoC into a Xilinx Kintex UltraScale KCU105 evaluation kit.

%\subsection{Simulation and verification}
\subsection{Validation}
%\subsubsection{Simulation example}
%A view extracted from the system simulation exposing the SafeDE working mechanism is found in Figure \ref{fig:chronogram}. First, at cycle 1, the chronograph shows how the lower threshold ($TH_{min}$) is crossed (instructions difference becomes smaller than 10), and the signal to stall the trail core (core2) is raised. The trail core remains stalled until the head core computes at least one instruction (just one cycle in this example). Later, at cycle 5, the upper threshold ($TH_{max}$) is crossed (instruction difference becomes larger than 15). Again, the head core (core1) is stalled until the trail core executes at least one instruction (until cycle 8 in this example).

%\subsubsection{Verification}
%To further evaluate the correct functioning of SafeDE once implemented in the FPGA, two registers that store the minimum and the maximum staggering that both cores reach during the execution are integrated. During the execution of the TACLe benchmarks set, in no case, the staggering is lower or higher than the values allowed by the thresholds.

%\begin{figure}[t!]
%\centering
%  \includegraphics[width=1\columnwidth]{imgs/chronogram.png} 
%  \caption{Piece of a chronogram extracted from a simulation of the system executing the TACLe benchmark Fac and with SafeDE active forcing a controlled diversity.}
%  \label{fig:chronogram}
%\end{figure}

In order to validate the correct functioning of SafeDE once implemented in the FPGA, we have added a register recording the lowest staggering observed between the head and tail cores. We have used the TACLeBench benchmark suite~\cite{taclebench}, which is a set of open-source self-contained benchmarks intended to evaluate basic functionalities in real-time systems. They have been chosen because, since their source files already include inputs hardcoded, they can be easily compiled and run on a baremetal setup without any support to read data from files. Moreover, since some of the benchmarks are quite simple (i.e. execution times range between some hundreds and some millions of cycles), they ease debugging and validation on a simulated environment.
Therefore, we have set the staggering to 10 cycles, $TH_{stag} = 20$, and recorded the lowest staggering observed across all benchmarks. Our experiments confirm that the actual staggering has never been below this number of cycles, hence providing evidence that SafeDE works as expected.


\subsection{Execution time overhead}


%\subsubsection{Execution time overhead in a set of TACLe benchmarks}
To elucidate the impact of SafeDE in terms of computational overhead, we have run the TACLeBench benchmarks in three different scenarios:
\begin{itemize}
\item \emph{Isolation}: only one core executes the benchmark and the other core remains idle. 
\item \emph{Redundancy without diversity}: two different cores execute the same benchmark without any control mechanism. 
\item \emph{Redundancy with diversity} enforced by SafeDE: SafeDE guarantees that the minimum staggering, $TH_{stag}$, is never exceeded. 
\end{itemize}

In our evaluation, we have set $TH_{stag} = 20$ for illustration purposes. Note that the lowest value that must be used for the staggering relates to the pipeline depth of the core (7 stages in the specific platform used) given that the instructions difference is obtained using committed instructions. Hence, using a pipelined core, it could occur that, by the time staggering is about to fall below the threshold and the trail core stalled (i.e. its commit stage is stalled), the pipeline of the trail core could be executing some common instructions to those of the head core in some of the stages if the staggering is too low. Thus, we have set the threshold to be high enough so that this cannot happen in a pipeline with 7 stages and a pipeline width of 2 instructions.

%Large instruction difference between thresholds limits the overhead because it is less likely for each core to be stalled. Since this is not a limitation for most applications, we chose a conservative difference. Concretely, SafeDE is configured with $TH_{min} = 20$ and $TH_{max} = 1000$.

Note that, by using a light-lockstep approach, redundant processes are generated loading binaries twice (one for each core) in different memory segments.
%Unlike the tight lockstep approach, both cores execute different memory segments containing the same benchmark, so both executions run isolated. 
We execute each benchmark 1,000 times and use the average cycle count for each one for our evaluation to discount the effect of small variations due to, for instance, delays caused by DRAM refreshes. In any case, absolute variations observed are always in the order of few tens of cycles.

\begin{figure}[t!]
\centering
  \includegraphics[width=1\columnwidth]{imgs/tacle_results.png} 
  \caption{Execution time of different TACLeBench benchmakrs normalized w.r.t. their execution time in isolation. Each benchmark is executed 1,000 times.}
  \label{fig:tacle_results}
\end{figure}

Results are shown in Figure~\ref{fig:tacle_results}. As shown, in all the cases, the execution time overhead with SafeDE w.r.t. the execution time in isolation and in two cores without SafeDE is negligible. In particular, SafeDE causes an execution time degradation in most of the cases below 0.5\%, and up to 1.3\% in one case (\texttt{BITONIC} benchmark) w.r.t. the execution time in isolation. If we compare it against the redundant execution without enforcing diversity, execution time degradation is generally below 0.1\% (in some cases performance even marginally improves), and up to 0.6\% for \texttt{IIR} benchmark.

%The biggest difference is found during the execution of the benchmark IIR and is smaller than a 0.6\%. The average overhead across all benchmarks w.r.t isolation execution is 0.17\%.

In some cases, minor performance variations between the isolation and redundant versions of the programs are observed, being those differences neither caused by interference between redundant threads, nor by SafeDE operation itself. Instead, those variations are caused by the initial core state (e.g. branch predictor state), or changes in instruction cache behavior due to changes in the memory alignment of the binaries with and without thread redundancy.
%effects related to instruction pipelining and memory alignment into instruction cache lines of the code may cause minor performance variations between the isolation and redundant versions of the programs not related to interference between redundant threads or SafeDE operation itself. 
In the case of \texttt{FAC} benchmark, since it is a small benchmark (around 700 instructions only), these tiny effects have a visible impact in relative terms (e.g. 1\% execution time increase without SafeDE and 0.1\% decrease with SafeDE). 

%As shown, in the case of \texttt{FAC} benchmark, the execution time with SafeDE is slightly lower than in isolation. Note that this behavior occurs systematically across the 1,000 runs performed for this benchmark. We have analyzed the source of this behavior in a simulated environment and found out that this benchmark is quite sensitive to the combination of pipelining effects and the state of the branch predictor. This benchmark, which is fact is a factorial program, performs frequent conditional branches. The first branches as well as the last ones may be missed. Hence, depending on how interference in the bus occurs, delays to access L2 cache (e.g. to write through data) may cause increased execution time or may avoid following mispredicted paths. This explains why uncontrolled diversity (without SafeDE) experiences interference in the bus that increases execution time, whereas controlled diversity (with SafeDE) avoids following some mispredicted paths. In any case, these are minor effects that would typically cancel out, but have an accumulative effect in our baremetal setup.

%We can see that for benchmarks Fac and Countnegative, the execution time happens to be smaller during the execution with SafeDE than in the execution in isolation. These results are explained due to a difference in the internal state of the core running in isolation and the cores running with SafeDE. This difference is caused both by synchronization mechanisms that have to be applied when two cores are active and by the SafeDE configuration previous to the execution. In these two benchmarks, this difference in the internal state affects the branch predictor behavior and causes systematically small differences in every iteration that increase the execution time in isolation w.r.t. the execution time with SafeDE. Since the overhead caused by SafeDE is minimal, the increase in the execution time due to a slower branch predictor in isolation is significant enough to end up being bigger than that overhead produced by SafeDE.

Overall, as expected, execution time increase can be regarded as negligible since the staggering threshold can be set very low (20 cycles in our evaluation), thus far below the 100$\mu$s, which correspond to many thousands of cycles, imposed by the software-only solution~\cite{SergiDFT}.


%\subsubsection{Relation between instruction difference between thresholds and execution time overhead}
%To infer how the instruction difference between both thresholds affects computational overhead, we have run the entire set of TACLe benchmarks, one after the other, 100 times and measured the total exectution time. We have done this 30 times and modifying the thresholds each time by keeping the $TH_{min}$ constant to 20 instructions, whereas $TH_{max}$ is increased. Time overhead for each pair $TH_{max}-TH_{min}$ can be seen in Figure \ref{fig:thresholds_overhead}. In the Y-axis is represented the time overhead of executing the entire TACLe benchmarks set with SafeDE w.r.t the execution time of the entire TACLe benchmarks set with only one core.

%\begin{figure}[t!]
%\centering
%  \includegraphics[width=1\columnwidth]{imgs/thresholds_overhead.png} 
%    \caption{Total execution time overhead with SafeDE of the whole set of TACLe benchmakrs w.r.t. its execution in isolation. For each run, the entire set of TACLe benchmarks are executed 100 times with the specified threshold configuration.}
%  \label{fig:thresholds_overhead}
%\end{figure}

%Overhead at low values of $TH_{max}-TH_{min}$ is above 45\%. However, overhead quickly decreases when the instruction difference between both thresholds increase. At 30 instruction distance, overhead is only 1\%, and at 170 instructions difference, overhead further reduces to 0.2\%. The more significant the difference between both thresholds, the smaller the overhead. Therefore, we can conclude that if SafeDE is used with an appropriate difference of instructions between both thresholds, the overhead becomes negligible.


\subsection{Hardware costs}

We synthesized our RISC-V design using the Vivado 2018 Toolchain and target the FPGA present in the Xilinx UltraScale KCU105. The overall cost of SafeDE implementation is 261 LUTs and 417 registers, whereas the entire SoC uses approximately 114,000 LUTs and 74,000 registers. Each core uses around 38,000 LUTs and 17,000 registers. Hence, SafeDE is a low-cost component representing just 0.23\% of the entire SoC LUTs and 0.56\% of entire SoC registers. SafeDE uses just 0.35\% of the LUTs of the pair of cores it manages, and 1.23\% of their registers. These numbers can be further improved by removing all the logic devoted to gather statistics. 


\section{Related Work}
\label{sec:rel}

\begin{table}[t!]
\centering
\caption{Classification of redundant execution techniques.}
\label{table:Fault-Tolerant-Classification}
\begin{tabular}{|c|c|c|l|}
\hline
Strategy & Target & Diversity & \multicolumn{1}{c|}{Approaches} \\ \hline
\multirow{5}{*}{HW} & \multirow{3}{*}{CPU} & Yes (tight) & ~\cite{STlockstep,Iturbe2019,relaxing_criticallity} \\ \cline{3-4}
 &  & Yes (light) & ~\textbf{Our approach (low staggering)} \\ \cline{3-4}
 &  & No & ~\cite{Rotenberg1999,SMTredundancy,Mukherjee2002,Gomaa2003,dynamic_coupled_cores,on_demand_redundancy} \\ \cline{2-4}
 & \multirow{2}{*}{GPU} & Partially & ~\cite{SergiDATE} \\ \cline{3-4}
 &  & No & ~\cite{instruction_replication_gpu,Jeon2012,Swapcodes,Nathan2015} \\ \hline
\multirow{4}{*}{SW-Only} & \multirow{3}{*}{CPU} & Yes (light) & ~\cite{SergiDFT} (high staggering)\\ \cline{3-4}
 &  & \multirow{2}{*}{No} & ~\cite{Haas2017,Shye2007,Scales2010,Reis2005}, \\ %\cline{2-4}
 &  &    & ~\cite{So2018,Alhakeem2015,Shye2009,Mushtaq2013} \\ \cline{2-4}
 & \multirow{2}{*}{GPU} & Partially & ~\cite{SergiIOLTS} \\ \cline{3-4}
 &  & No & ~\cite{softwareapproaches_gpgpureliability,inter-intra-workgroup,Jain2019,Vargas2018} \\ \hline
\end{tabular}
\end{table}

Some works investigate redundancy for CPUs, yet without diversity including Redundant Multi-Threading in a core~\cite{SMTredundancy,Rotenberg1999}, across different cores~\cite{Mukherjee2002,Gomaa2003,dynamic_coupled_cores}, and providing only partial redundancy~\cite{relaxing_criticallity,on_demand_redundancy}. Those works lack diversity by reexecuting on the same hardware, or using different cores without staggering.
Software-only solutions for CPUs build on the compiler to enforce redundancy creating monitoring threads or resorting to transactional memory~\cite{Reis2005,So2018,Haas2017,Mushtaq2013,Shye2007,Shye2009}. Unfortunately, at least CCFs affecting redundant computing units are not covered by those solutions.

Some works targeting GPUs provide hardware support for redundancy~\cite{inter-intra-workgroup,Jeon2012,Swapcodes,Nathan2015} or software-only support~\cite{softwareapproaches_gpgpureliability,inter-intra-workgroup,Jain2019}, but they fail to guarantee diversity. Diverse redundancy on GPUs has been proven doable with~\cite{SergiDATE} and without hardware support~\cite{SergiIOLTS}. Note that those solutions are GPU-specific, so cannot be applied to CPUs.

As discussed before, some processors implement tight lockstep in the form of DMR (Infineon AURIX processor family~\cite{infineon_aurix}, ST Microelectronics SPC56XL70~\cite{STlockstep}), or TMR (Arm Cortex-R5 based designs~\cite{Iturbe2019,paper1}).
Other works focus on how to expose latent errors to shorten detection time~\cite{TCAD}, and how to enhance error recovery~\cite{DATEcheckpointing}.
Finally, Reviriego et al.~\cite{DivDMR} focus on how to perform recovery efficiently for DCLS (diverse DMR) designs.
However, as detailed before, tight lockstepping halves (for DMR) the number of user-visible cores, which impacts flexibility and, ultimately, performance.

Light-weight lockstepping for CPUs has been addressed with software-only solutions so far~\cite{SergiDFT}, but staggering is significant and an additional core is needed to run the monitor. Our solution, SafeDE, drastically reduces staggering and removes the need for an additional core at the expense of introducing a lowly intrusive hardware module. Table~\ref{table:Fault-Tolerant-Classification} summarizes related work and puts SafeDE in context.
\section{Conclusions}
\label{sec:concl}


Safety-related systems must implement diverse redundancy for the highest integrity functionalities to avoid CCFs. Tight lockstepping is the \emph{de facto} solution for CPUs, but it makes half of the cores not visible at user level, so significant performance is lost when none or few high-integrity tasks are run. Light-weight lockstepping has been proposed recently to overcome such limitation and gain flexibility. However, existing solutions build on slow software feedback loops that impose large staggering and require an additional core to run the monitoring process.

This paper introduces SafeDE, a tiny module implementing light-weight lockstepping with a very short feedback loop (e.g., 20 cycles), hence causing negligible performance impact, and not needing any additional core since SafeDE itself controls the feedback loop. Our results show that SafeDE incurs both negligible performance degradation (0.5\% on average) and hardware overheads ($\approx$0.5\% extra SoC area) w.r.t. to a non-redundant industrial SoC, and effectively captures all CCFs that would also be captured by tight lockstep execution.


\section*{Acknowledgements}
This work has received funding from the European Union's Horizon 2020 research and innovation programme under grant agreement no. 871467. 
This work has also been partially supported by the Spanish Ministry of Science and Innovation under grant PID2019-107255GB.


\bibliographystyle{plain}
\bibliography{biblio}

\end{document}
