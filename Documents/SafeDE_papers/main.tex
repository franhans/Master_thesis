\documentclass[conference]{IEEEtran}
\IEEEoverridecommandlockouts
% The preceding line is only needed to identify funding in the first footnote. If that is unneeded, please comment it out.
\usepackage{cite}
\usepackage{amsmath,amssymb,amsfonts}
\usepackage{algorithmic}
\usepackage{graphicx}
\usepackage{textcomp}
\usepackage{xcolor}
\def\BibTeX{{\rm B\kern-.05em{\sc i\kern-.025em b}\kern-.08em
    T\kern-.1667em\lower.7ex\hbox{E}\kern-.125emX}}
\begin{document}

\title{Light-weight lockstep processors for increased performance and safety
\thanks{\color{red}This research was founded by.....}
}

\author{\IEEEauthorblockN{Francisco Bas, Felipe Gallego Ruben Lorenzo, Guillem Cabo, David Trilla, Jaume Abella}
\IEEEauthorblockA{Barcelona Supercomputing Center}}

\maketitle

\begin{abstract}
Functional safety is one of the most important (if not the most) design consideration for Safety-Critical Systems (SCS).
Safety standards (e.g. ISO-26262 for automotive) define what measures must be taken in specific instances of SCS' to ensure that critical hazards
are sufficiently mitigated. At the highest levels of safety (e.g. ASIL-D in automotive, DAL-A for avionics) systems are required
to keep operating even under the presence of faults, which often involves the redundant execution of task with double or even triple redundancy schemes
depending on the amount of safety required.
However such replication incurs in non-neglegible area costs, sometimes devouting cores to perform redundant work,
hence diminishing amount of transistors that can be devouted to obtaning more performance.
In this paper we present a modular light-weight lockstep execution unit, that provides lockstep support
on demand to multicore architectures, therefore adding flexibility and enabling high-performance or safe
execution whenever needed. Our module attaches to commonly available core counter and control signals
avoiding costly reverificaiton of already existing Functional Unit Blocks (FUBs) and providing ease of
scalability. We show that our proposal only incurs in X\% penalty while being able to detect a high amount of faults.
\end{abstract}

\begin{IEEEkeywords}
functional safety, redundancy, diversity, faults
\end{IEEEkeywords}


\section{Introduction}


The current embedded domain is being transformed by the rise of new applications and current market trends to AI applications not only into cloud domains but also at the edge devices. Some of these edge devices will inevitably perform some kind of critical task. For instance, autonomous driving already requires AI to perform critical task whose failer can cause harm. These subset of embedded systems, also called Safety-Critical Systems, are therefore required to provide the desired performance to run such applications while maintaining the safety standards dictated by rules like ISO-26262~\cite{ISO26262} in automotive or DO-178B/C~\cite{DO178BC} in avionics.



Under such conditions, MultiProcessor System-on-Chip (MPSoC) have risen as a solution to provide teh needed performance and embedded capabilities while in a reduced form factor.

\section{Background \& Related Work}
\label{sec:back}

\section{Evaluation}
\label{sec:real}
\cite{NGMP}

\section{Conclusions}
\label{sec:conc}



\bibliographystyle{plain}
\bibliography{biblio}

\end{document}
