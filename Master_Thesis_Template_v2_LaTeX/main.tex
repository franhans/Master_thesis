\documentclass[a4paper,12pt]{article}

\input{tools/packages.tex}
\input{tools/styles.tex}
\input{tools/acro.tex}

\begin{document}

%%% COVER %%%
\fancypagestyle{alim}{\fancyhf{}\renewcommand{\headrulewidth}{0pt}
\cfoot{\includegraphics[height=2.2cm]{img/logos/logo_telecos.png}}
}
\thispagestyle{empty}
\begin{center}
{\sffamily 
\resizebox{0.8\textwidth}{!}{\includegraphics{img/logos/upc_completo+telecos.png}}\\
\vspace{1cm}
{\Huge Design of a diversity enforcement module for safety critical processing systems}\\
\vspace{0.5cm}
{\color{black}\hrule height 1pt}
\vspace{1cm}
{\large{Master Thesis\\
submitted to the Faculty of the \\
Escola T\`ecnica d'Enginyeria de Telecomunicaci\'o de Barcelona \\
Universitat Polit\`ecnica de Catalunya \\
by\\
\vspace{0.4cm}
Francisco Bas Jalón}}

\nocite{*} % only if you no use \cite{}

\vspace{1.5cm}

{In partial fulfillment\\
of the requirements for the master in\\
\textit{Electronic} \textbf{ENGINEERING}}

\vspace{2cm}

{{Advisor: Francesc Moll Echeto}} \\
{{Advisor: Jaume Abella Ferrer}} \\
{{Barcelona, Date 02/07/2022}}
\thispagestyle{alim}
}

%%% INDEX %%%
\end{center}
\newpage
\tableofcontents

%%% LISTS %%%
\newpage
\listoffigures
\lstlistoflistings
\listoftables

%%% REVISION %%%
\newpage
\input{revision_history}

%%% ABSTRACT %%%
\clearpage
\newpage
\section*{Abstract}

Safety-critical systems must adhere to specific functional safety standards describing the development process for those systems. One key requirement is the ability to avoid a single fault from causing a system failure, or in other words, avoiding Common Cause Failures (CCFs). Redundancy is a usual solution against CCFs. However, some specific CCFs may affect redundant components identically (e.g., voltage droops, clock interferences), hence potentially leading to identical errors that may go unnoticed and cause a failure.
Diversity is often deployed along with redundancy to avoid also those CCFs. In the particular case of computing elements (e.g., cores), this is usually realized with some form of lockstep execution where two identical cores execute the same software, but with some time shift among them (aka staggering). Therefore, both cores have different state at any point in time and faults affecting both cores lead to different errors, which can be detected by comparing the outputs.
Unfortunately, existing solutions have some non-negligible costs: (i) hardware-only solutions hide half of the cores making them non-user visible, hence halving platform performance even for non-critical tasks. Conversely, (ii) software-only solutions are much more flexible but impose the use of a third core to run the lockstep monitor, and require large staggering which has significant impact in performance for short programs.

This thesis devises a new solution aiming at combining the advantages of existing solutions. Our proposal, a hardware diversity-enforcement module (referred to as SafeDE), is an efficient hardware realization of the software monitor. Therefore, it does not hide any core to the end user, it does not require a third core for monitoring purposes, and allows operating with tiny staggering (e.g., few tens of cycles instead of hundreds of thousands as required for the software-only solution). 

We implement and integrate SafeDE in a space multicore prototype in an FPGA and validate that it effectively achieves its requirements with negligible hardware costs. Moreover, this work has already led to the publication of two peer-reviewed articles in especialized conferences and journals.

%{Every copy of the thesis must have an abstract. An abstract must provide a concise summary of the thesis. In style, the
%abstract should be a miniature version of the thesis: short introduction, a summary of the results, conclusions or main
%arguments presented in the thesis. The abstract may not exceed 150 words for a Degree’s thesis.}

\clearpage
\newpage
\section{Introduction}

{An Introduction that clearly states the rationale of the thesis that includes:}

\begin{enumerate}
\item {Statement of purpose (objectives).}
\item {Requirements and specifications.}
\item {Methods and procedures, citing if this work is a continuation of another project or it uses applications, algorithms,
software or hardware previously developed by other authors.}
\item {Work plan with tasks, milestones and a Gantt diagram.}
\item {Description of the deviations from the initial plan and incidences that may have occurred. }
\end{enumerate}

\bigskip




%%% StateOfTheArt %%%
\clearpage
\newpage
\section{Background and state of the art}

\subsection{Faults, Failures and Errors}

During this thesis, the common terminology in fault-tolerant systems \cite{AlgirdasAvizienis2004} is employed:

Any electronic system delivers a service that the user of that system perceives. This service comprises all the external states of the system. A service \textbf{failure} or \textbf{system failure} occurs when the delivered service (i.e., one or more external states) deviates from the correct service state. The correct service is defined by the functional specification of the system. A failure in safety-critical systems can endanger lives or produce high economic losses. Thus, the main goal of safety-critical systems is to minimize the probability of a system failure.

The deviation between the correct internal or external state and the real state is called an \textbf{error}. The cause of an error is called a fault. Thus, a \textbf{fault} is a defect within the system. A fault first causes an error in one of the components that form the system, altering the system's internal state. If this error propagates to the system's output altering the external state and the service provided, we will say that the error led the system to a failure. However, not all faults produce errors and not all the errors reach the external estate of the system producing a failure.  

For instance, consider a two-inputs AND gate inside a system. If one input of the gate is '1' and the other is '0', the expected output will also be '0'. In this scenario, a fault that flips the logic driving the input with value '0' to a '1' will modify the AND gate output to '1', producing an error. However, if a fault flips the other input from '1' to '0', the output will still be '0', the expected value. 

Following the same logic, an AND gate, whose inputs are driven from two registers, one of them with an incorrect value (error), could correct the error preventing it from spreading to other registers and reaching the output of the system.

Faults can be classified into two main categories: \textbf{Systematic faults} that are related in a deterministic way to a certain cause and are avoidable by construction. These faults can be avoided by taking into account possible faults during the first step of the design or by investing enough resources into verification and validation processes \textcolor{red}{(examples....)}. \textbf{Random faults} that occur unpredictably following a probabilistic distribution and are unavoidable. This work focuses on Common Cause Faults (CCF), a particular type of random faults that will be explained later. \textcolor{red}{(examples....)}

\bigskip


\subsection{Safety Related Systems}

Safety-critical systems are those systems that need to work correctly because otherwise, a failure or malfunctioning could jeopardize people's life or health, produce losses in expensive equipment or cause environmental harm. For this reason, these systems must have mechanisms to lower the failure rates until they happen with a negligible likelihood. For instance, in the standards of the aircraft industry, an acceptable failure rate is $10^-9$ accidents per hour \cite{bowen2000ethics}.

Some errors, like systematic errors, can be found and corrected during the development process or mitigated by applying qualitative measures depending on the desired system integrity level SIL \textcolor{red}{citar libro??}. However, random faults could not be avoided and require special mechanisms to prevent these faults from producing a system failure or at least to minimize the likelihood of these happening until a reasonable stent.  

Faults can also be classified into permanent, intermittent and transient faults \cite{constantinescu2003trends}: Permanent faults produce irreversible physical changes to the hardware. Intermittent faults are those faults that appear in irregular intervals. Transient faults occur because of temporary environmental conditions. 

Transient faults are also called soft errors. These faults alter the normal behavior of the system momentaneously. Transient faults produce a loss of data, but they do not produce any physical damage to the circuit. They are random by nature and can appear at any time in some parts of the system, causing a deviation from the expected behavior. Several sources of transient faults exist: neutron and alpha particles, power supply variations and interconnect noise, electromagnetic interference and electrostatic discharge. These sources can affect one or several transistors, momentarily modifying their behavior. Loss of reliability in digital safety-critical systems is produced mainly by transient faults \textcolor{red}{try to add pages to the cite} \cite{enso2003fault}. 

According to Moore's Law, the number of transistors that fit in the same area increase by a factor of two every year. The industry has followed this trend for many years, and even though the trend is slowing down, transistors are smaller every year. Increasing the number of transistors in a system also increases the probability of any of them experiencing a soft error. Besides, lower power voltages, higher frequencies and shrinking transistors geometries make them more vulnerable to other sources of faults. For instance, higher frequencies and smaller interconnect features increase the possibility of violating the timing safety margins of the system. Also, lower voltages along with smaller transistors make systems more vulnerable to neutron and alpha particles \cite{constantinescu2003trends}. Hence, this trend has a negative impact on systems reliability. 

Also, with smaller transistors, variations in the manufacturing process (in widths, lengths, oxide thickness, etc.) are more likely to produce systematic failures in the integrated circuits. 

Safety-critical systems must operate correctly even in the presence of faults. The systems that integrate mechanisms to allow a correct operation even when faults appear are called fault-tolerant. Fault-tolerant systems must include two basic mechanisms: fault detection and recovery.

The system must be equipped with components able to detect the errors and prevent them from propagating to other components. When the error is spotted, these components alert the system to trigger a recovery mechanism that puts the system in a previous error-free state.

When the system detects an error, it can recover the last known error-free state, reset the system by powering off or enter a safe state mode, for example, in the event of a permanent fault. However, detecting a transient fault in our system is not always easy. Resetting the system is not always possible and recovering the last error-free state requires a mechanism to store those states. 

\bigskip



\subsection{Redundancy}

\textcolor{red}{make a better definition of redundancy in ECC}
\textcolor{red}{Add an aclaratory image}

Errors are usually detected by employing redundancy. Redundancy is applied differently for different parts of the system. For instance, Error Correction Codes (ECC) is employed to protect the stored data \cite{alcaide2019software}. The data is encoded whit redundant information which allows for detecting errors. In the case of the computing elements, two different kinds of diversity are employed: Time redundancy and space redundancy.

When \textbf{time redundancy} is applied, the same operation (an instruction or a set of instructions) is executed several times (more than once) in the same processing unit. On the other hand, \textbf{space redundancy} is achieved by replicating several times a given processing unit that performs the same operation. This approach is called Dual Modular Redundancy (DMR) \cite{gomaa2003transient}, \cite{lafrieda2007utilizing}, \cite{mukherjee2002detailed} when two core replicas exits on the sytem and Triple Modular Redundancy (TMR) \cite{iturbe2019arm} when the number of replicas is trhee. In a free-error execution, all the outputs must coincide. Therefore, by comparing the different generated outputs, either by the different executions or in the different processing units, the system is capable of detecting possible errors. 

When the different outputs do not coincide, the system has to activate a recovery mechanism to restore the system to a safe state and re-execute from there. This safety state can be a previous free-error state stored in memory or could be achieved by dropping the task if the time constraints allow it. For instance, a system executing a task with a small period (e.g., every 50 ms), such as braking and steering, must perform the task before its Fault Tolerant Time Interval (FTTI) (e.g., 200 ms). In this example, the FTTI is big enough w.r.t the task period to allow the system to drop the task and execute it all over again as long as two consecutive faults do not occur.

Time redundancy is effective against transient faults since it is highly improbable that a transient fault affects two consecutive executions in the same way, and the free-error execution output and the erroneous output will differ, allowing the comparator to detect the error. However, a permanent fault will cause the same error in both executions, making it impossible for the comparator to detect the error. On the other hand, space redundancy is more suitable to effectively detect permanent faults since it is likely that a systematic fault affects only one of the replicas of the hardware. 

Space redundancy has a significant area penalty because of the replication of processing units, but the performance loss is negligible since all the operations are performed in parallel. On the other hand, time redundancy has no area overhead, but the performance degradation is high since the same operation has to be performed several times sequentially.

Redundancy is based on the idea that the probability that all the replicas produce the same error is near zero. This assumption is valid in most of the cases since most of the faults are independent. However, if the same cause is responsible for different faults in the different replicas, faults are not independent anymore and the previous assumption is not valid.

\bigskip


\subsection{Sphere of Replication}
\label{section:SoR}

\textcolor{red}{add image from paper}

The Sphere of Replication (SoR) \cite{hernandez2014live}, is the granularity at which the outputs of the replicated elements are compared to detect errors. In \textcolor{red}{figure []}, several granularity levels are illustrated. They range from the finest granularity (a) to the most coarse one (d). Comparing data between pipeline stages (\textcolor{red}{add sub-image reference} is not practical due to the enormous hardware and performance overhead required to perform the comparison each cycle. Instruction granularity can be applied as shown in \cite{reinhardt2000transient}, where the authors propose executing redundant threads in a superscalar microprocessor with the capacity to execute several threads in different functional units. However, this approach is not exempt from drawbacks since special communication channels are required to perform the comparison at instruction-level granularity. On the other hand, a finer granularity means faster error detection and faster and easier recovery mechanisms. 

However, most of the approaches used in the industry rely on an off-core-level SoR. In this case, only the data written to the system memory or the I/O interface is compared. This approach takes advantage of the fact that a task finishes without errors when the service provided (state of the I/O and memory) is correct, and provided that the external states are correct, the internal states (e.g., in-core activity) can be ignored. Also, the overhead is much lower than the instruction-level SoR overhead. With off-core-level SoR, only the addresses and values sent by the cores through the communication-network are needed to perform the comparison. Hence, snooping the information flowing through the communication-network is enough and the intrusiveness of this solution is much lower. However, the time elapsed from a fault occurrence and the error detection is unbound. For instance, an error could be confined to the register file during most of the program execution time without reaching the communication-network. This is an issue for safety-critical real-time systems where each task has strong time constraints.

\bigskip





\subsection{Dependent failures and Common Cause Fauilures}

Dependent failures \cite{Tummeltshammer2009} are characterized by their occurrence probability. Whereas the occurrence probability of several failure events produced by independent faults can be computed as the product of all their occurrence probabilities, the occurrence probability of dependent failures can not be modeled in the same way. Hence, the following expression applies to two independent failures A and B, but is not valid if failures A and B are dependent. \[P(A) P(B|A) = P(B) P(A|B) = P(A) P(B)\]

Usually, as shown in the expresion below, the probability of two dependent failures is higher than that of two independent failures caused by independent faults. \[P(A) P(B|A) > P(A) P(B)\]

This issue has to be considered since redundancy is based on the assumption that the likelihood of two replicas experiencing the same failure is virtually zero.

CCFs are a type of dependent failures that arise simultaneously in redundant elements from a single shared cause. In safety-critical systems implementing space redundancy, CCFs may produce the same failure in all the replicas. If this happens, the error detection mechanism will fail since both erroneous outputs coincide. For this reason, CCFs are a hazard for safety-critical systems and all fault-tolerant systems standards consider CCFs' effects.

For a CCF to occur, the system must have at least two channels (two replicated elements). A CCF arises when a fault that is the root cause spreads through a coupling mechanism to all the system channels, \textcolor{red}{add figure and reference}. For instance, suppose a fault-tolerant two-channel system (i.e., two redundant cores in the same die) where the cooling system fails. The root cause is the failure of the cooling system. The heat is not dissipated anymore, and the temperature in the whole die increases. In this example, the thermal coupling becomes the mechanism responsible for the fault cause affecting both channels.

Notice that both systematic and random faults can also cause a CCF. For instance, typical examples are a soft error affecting the clock logic of the two cores or a voltage droop. Defects on the design of the hardware of the replicas or in the manufacturing process (e.g., identical physically low gates in the replicas) could affect all the channels in the same way producing a CCF.

Figure \textcolor{red}{reference the image again} shows the most relevant fault coupling mechanisms \cite{Tummeltshammer2009}, which are described in the follosing paragraphs. 

\textbf{Coupling by similar design or fabrication process:} Usually same software and hardware designs are employed for all of the system channels (e.g., all the cores have the same design and the software they execute is also the same). Employing different hardware or software for each channel will dramatically increase the design and test costs. Hence, a fault in the design or the manufacturing process must be prevented during the design process or detected during the testing. Otherwise, it will be missed by the fault-detection mechanism even if redundancy is applied. 

\textbf{Mechanical and Thermal coupling:} The mechanical and thermal coupling effects are transmitted slowly over the SoC die compared to the processor operation speed. It is assumed that a CCF can only arise if the effects of the mechanical stress or the heating affect the same gate/transistor of all of the replicas at the same time. However, this is very unlikely due to the slow propagation of these physical effects. The most hazardous scenario would consist of a short circuit producing a very focused and abrupt temperature increase. This could result in a CCF if the affected point presents a perfect symmetry w.r.t the affected gates/transistors. It is assumed that mechanical stress affects the whole die, and therefore, it does not represent a CCF risk. 

\textbf{Electromagnetic coupling:} Electromagnetic coupling can affect the layout when the paths of the cores act as an antenna for the electromagnetic field. This could result in voltage changes producing soft errors in both replicas. If the layout of all the replicas is identical, the effect of the electromagnetic fields is very likely to be similar in all the system channels. However, the small dimensions of the VLSI protect them against frequencies below 100GHz, since the circuit paths only work as antennas for higher frequencies. PCB traces are much more vulnerable to electromagnetic fields. Therefore, it is improbable that electromagnetic fields produce a CCF in an integrated circuit.

\textbf{Electrical coupling:} Usually, both cores share the power supply and some signals such as the clock. Therefore, a perturbation in the power supply, e.g., voltage droop or noise or a soft error in the clock logic, can affect all the replicas of the system analogously. Unlike mechanical or thermal coupling, electrical coupling affects the whole system concurrently, and hence, in case of similar effects, a CCF that can not be detected by the comparator unit of the fault detection mechanism will occur. Thus, diversity is not enough to prevent system failures when a fault affects the different replicas in the same way and extra measures must be applied. 

Notice that even though single event upsets (SEUs) caused by particle radiation are one of the major causes of failures in electric systems [14], they usually affect a tiny area (usually a single register or cell RAM), making it very unlikely for them to produce a CCF.   

To protect the system against CCF and comply with the safety standards, the redundant elements in the system must show diversity among them. Thus, diverse redundancy can be achieved in different ways: 

\textbf{Design diversity} consists of achieving hardware procuring the same functionality but implemented in a different way. Design diversity can be achieved at different abstraction levels. For instance, the same functionality can be performed with two different architectures, or the same architecture can be implemented using different gate libraries or different technology. Design diversity is very effective in preventing transient and systematic faults from causing a CCF since it is improbable that any fault affects all the replicas of the system analogously. However, implementing two diverse components is associated with a considerable increase in design costs.

\textbf{Time diversity} is reached by ensuring that the redundant software executions in the replicas are staggered, i.e., the redundant cores are always executing different instructions, and thus their internal states are always different. Being the internal states different, a transient fault affecting all the cores analogously will produce different results and thus, the errors will be detected by the fault-detection mechanism. Applying time diversity implies lower costs and design efforts. However, time diversity is vulnerable against systematic faults that affect all the replicas analogously. %This approach, if realized with two cores, is referred to as Dual Core LockStepping (DCLS) and is used by several commercial processors [15], [35], [37].


\bigskip


\subsection{Lockstep execution}

The lockstep architecture implements two identical processors that perform a staggered execution of the same software (i.e., one core is always some instructions ahead of the other core). Lockstep execution provides a mechanism to detect CCF by implementing diverse redundancy in the system. The next section shows two different approaches for achieving a lockstepped execution.

\subsubsection{Tight hardware-based lockstep execution}

This approach is broadly used. For instance, it is implemented in the Infineon AURIX family \cite{infineon2012aurix}. In a hardware-based lockstep implementation, two identical instances of the same processor are employed (DMR). Both processors execute the same instructions with a small-time shift of N cycles (usually 2 or 3 cycles). The outputs of both processors are compared by hardware, detecting any possible mismatch between both executions. Also, since both cores have some cycles of staggering by design, the system exhibits time diversity, and hence it is protected against CCFs. 

Both cores assume a different role in this technique. The core that goes ahead in the program execution will be the head core, while the core that is behind in the execution will be the trail core. As shown in the figure[\textcolor{red}{figura lockstep}], the inputs are the same for both cores, although the trail core needs to queue the inputs in a buffer for N cycles. The head core's external requests (data load/store, interrupts, etc.) are stored in a buffer during N cycles until the trail core performs the same requests. Requests of both cores are compared before leaving the core complex. The error is detected in the event of a mismatch, and the proper recovery mechanisms can be applied.

As mentioned in section \ref{section:SoR}, errors can only be detected once they leave the core complex and are visible at the outputs. Therefore, any fault can go undetected for an arbitrary number of cycles that depends only on the workload. Some mechanisms can be applied to reduce the error time detection as described in \cite{hernandez2014live}. In this work, a mechanism to periodically send the file registers values through the system communication-network to detect errors before they reach the ouputs of the cores is proposed.

This lockstep implementation hides its complexity to the user, who perceives everything as one single processor, easing the development process. However, for the same reason, performance is halved since both processors can not be used to execute different non-critical tasks that do not require lockstepped execution. At the same time, this approach is very intrusive and significant hardware modifications are required to implement this solution in a couple of cores.

\subsubsection{Light-weight software-based lockstep execution}
\label{section:software_light_lockstep}

\textcolor{red}{add figure and revise references to it}
%a given program (task) twice on different cores [3]. Those tasks
This approach is proposed in \cite{alcaide2020software}. The idea is to create diverse redundancy at the software level by running a program twice on different cores. To implement this approach, no hardware modifications are needed. Thus, the intrusiveness of this approach is null in hardware terms and can be implemented with Commerical off-the-shelf (COTS) products.

Light-weight software-based lockstep execution requires three cores: one for monitoring the execution of the critical tasks and ensuring that the minimum required staggering between the redundant executions is maintained, and two identical cores to perform the redundant execution of the critical task. The monitor core executes the monitor thread, which is in charge of scheduling the redundant processes in the two different cores and periodically obtaining the number of executed instructions by each redundant core ($\#instr$ in the figure). The monitor only allows the trail core to make progress if staggering ($\#instr\_head - \#instr\_trail$) is bigger than a given threshold $TH\_stag$. This condition is checked by the monitor every $T\_check$ cycles. $TH\_stag$ and $T$ have to be selected such as if the trail core executes in T the maximum possible number of instructions while the head core is stalled, the staggering still is bigger than zero. Only when the head core finishes the execution of the critical task the monitor allows the trail core to run without exercising any control over it. 

The monitor thread is in charge of keeping a proper staggering between both redundant executions to procure time diversity. However, the monitor does not provide a mechanism to detect faults during the execution. This checking mechanism is implemented by software means and when the execution finishes, its results are compared between cores.

The core executing the monitor thread is unprotected against CCFs. Therefore, its execution requires a hardware-based lockstepping. The thread monitor is also in charge of handling the inputs and the outputs of the system. Two input readings from both redundant processors at different times could result in different read values, which would make the execution results differ.  

This approach achieves a lockstepped execution with null intrusiveness in hardware terms. However, there is a trade-off between the period T\_check and the minimum staggering allowed $TH\_stag$. The smaller the $T\_check$ period, the more significant the thread monitor computational overhead. Also, the smaller $T\_check$ period is, the trail core can execute fewer instructions in that period, and the $TH\_stag$ is smaller. This trade-off is usually resolved by choosing a staggering equal to the number of instructions the core can execute in 100$\mu$s. Therefore, the biggest drawback of this approach is its high staggering.

Note that with the light-weight approach, both cores can execute the same redundant tasks when the criticality level is high, ensuring a safe execution or two different non-critical tasks when lockstepped execution is not required. In the last case, the light-weight approach doubles the performance.

\bigskip


\subsection{Triple Modular Redundancy???}

%\clearpage\section{State of the art of the technology used or applied in this thesis:}
%
%{A background, comprehensive review of the literature is required. This is known as the Review of Literature and should
%include relevant, recent research that has been done on the subject matter.}
%
%\subsection{Topic}
%
%Here you have a couple of references about LaTeX ~\cite{latexcompanion} and electrodynamics \cite{einstein}.
%
%\bigskip
%
%\subsection{Topic}

%%% METHODOLOGY %%%
\clearpage
\newpage
\section{SafeDE concept}

\textcolor{red}{rewrite this}
This section presents the architecture of SafeDE, its features and limitations, its extension towards N-modular redundancy, and its implementation and integration (both hardware and software) details.

\bigskip


\subsection{Architecture}

\textcolor{red}{add image, check image names}

SafeDE is built on the light-weight lockstepping concept. As explained in the section \ref{section:software_light_lockstep}, this approach has already been implemented in software. However, the implemented software light-weight lockstep needs a third core running the monitor thread and the frequency at which the monitor can retrieve the executed instruction, compute the staggering and apply corrective measures is low. As a consequence, the long feedback loop imposes a large staggering. 

SafeDE is a tiny hardware module that monitors the execution of the redundant cores imposing some staggering to reach time diversity. As shown in Figure \textcolor{red}{imagen referncia} SafeDE collects every cycle the instructions executed by the two cores ($\#instr\_head and$ $\#instr\_trail$) and computes the staggering that exists between both cores ($instr\_head - instr\_trail$). If the head core is not at least $TH\_stag$ instructions ahead of the trail core, SafeDE will raise the trail core stall signal that freezes its pipeline registers (registers keep the same value). The stall signal will be set to zero whenever the head core makes enough progress and the staggering is again bigger than $TH\_stag$ instructions.

$TH\_stag$ is the minimum staggering (in terms of instructions number) that SafeDE enforces between both cores. Typically, $TH\_stag$ has a low value. For instance, a value bigger than the pipeline stages can be selected to ensure that the content of both pipelines is always completely different. Hence, the staggering enforced by SafeDE is much smaller than that enforced by the software approach, and it is comparable to the hardware-based lockstep execution staggering. Since SafeDE is implemented directly on hardware is capable of computing the staggering and stalling any of the cores every cycle, overcoming the light-weight lockstepping limitations while maintaining its advantages.

As in the light-weight software-based lockstep execution, a software mechanism must be implemented to compare the results of the executions once they finish. 

Notice that the execution of the redundant critical tasks has to be independent. Each critical task has to be allocated in a different memory space address for that purpose. SafeDE is controlled by means of internal registers. Each core has to configure SafeDE once it reaches the critical section that needs lockstepped execution. SafeDE configuration is performed using the corresponding driver. An \textcolor{red}{API/auxiliar software?} is also needed to schedule both redundant processes to the corresponding cores in case the tasks are running on top of an operative system. Later in section \ref{section:software_integration} these software components are described both in the context of bare metal and Linux integrations.

Note that neither core has a pre-defined head or trail core role. The first core indicating SafeDE that has reached the section needing lockstepped execution assumes the head core roll while the other core assumes the trail roll.


\subsection{SafeDE Implementantion and Integration}

We have integrated SafeDE in two different multiprocessor platforms based on CAES Gaisler RISC-V NOEL-V cores. This section describes both platforms and provides detailed information on the SafeDE hardware implementation and integration. Later, the SafeDE configuration process is explained. Finally, SafeDE software integration is reviewed both for a bare metal setup and for a platform running Linux.    

\bigskip


\subsubsection{De-RISC and SELENE Platforms}
We have integrated SafeDE in two different multiprocessor platforms based on CAES Gaisler RISC-V NOEL-V cores. This section describes both platforms and provides detailed information on the SafeDE hardware implementation and integration. Later, the SafeDE configuration process is explained. Finally, SafeDE software integration is reviewed both for a bare metal setup and for a platform running Linux.    

\subsection{De-RISC and SELENE Platforms}
\textcolor{red}{Maybe add an couple of images of the platforms}


\textbf{De-RISC platform:} The De-RISC platform \cite{gomez2020risc} is developed in the scope of a European project motivated by the lack of a high-performance Multiprocessor System on a chip (MSPSoC) suitable for space applications. Most of the existing platforms do not supply the necessary performance required by spacecraft, are not reliable enough and are not compliant with the safety requirements for space applications or face export restrictions like the use of proprietary Instruction Set Architecture (ISA). 

The project tries to overcome these limitations by adopting multicore processors in the space domain that provide the required performance but face some challenges related to space safety regulations, predictability and reliability. To avoid export restrictions due to proprietary ISAs, the platform is based on the open-source RISC-V ISA. 

De-RISC platform is composed of different reusable IP cores developed by CAES Gaisler. Those IP cores are contained in the GRLIB IP library, an integrated set of reusable IP cores designed for SoC development. The IP cores have a common on-chip bus interface. SafeDE has been added to the GRLIB IP library as another reusable IP core. The library includes cores for AMBA AHB/APB control. The library is provided under the GNU GPL license.

As a proof of concept, we have integrated SafeDE into the De-RISC industrial space platform. The platform is configured to instantiate 2 CAES Gaisler's NOEL-V 64-bit cores. NOEL-V cores implement the RISC-V ISA, are dual-issue, and implement a 7-stages pipeline. Both cores have one private L1 data and instructions caches. The IP cores are centered around several AMBA AHB and APB buses. Cores are connected to the L2 cache and some peripherals through a 128-bit AMBA Advanced High-performance Bus (AHB). A 32-bit low-bandwidth Advanced Peripheral Bus (APB) is instantiated for components requiring low bandwidth and is connected to the main AHB Bus through an AHB/APB bridge. The L2 cache is shared for all the cores and is connected to the Memory controller through another 128-bit AHB bus.


\textcolor{red}{citar repo SELENE}
\textbf{SELENE platform:} SELENE \cite{SELENEgit} is another European project that focuses on developing a High-Performance Computing (HPC) Multicore platform capable of delivering the computation capabilities needed by autonomous systems in safety-critical domains such as space, avionics, robotics and factory automation. HPC platforms impose some difficulties when being certified for safety-critical systems since they usually lack support for functional and timing isolation and testability. The project tries to overcome these limitations by developing an open-source Safety-critical Cognitive Computing Platform (CCP) with self-awareness, self-adapting, and autonomous capabilities. 

SELENE computing platform builds upon a combination of multicore and accelerators that will be prototyped on an FPGA so that they can be extended and upgraded. SELENE platform is also developed using the IP CORES from GRLIB IP CAES Gaisler and other GPL IP cores developed by other project partners. SafeDE is one of the IP cores devised to be integrated into the SELENE platform to make it safer.

SELENE platform is very similar to the De-RISC platform since the basic IP cores used to build the platform are implemented from the GRLIB IP library. However, the SELENE platform offers more complexity since it instantiates more NOEL-V cores and shared L2 cache interconnects with a Network on a Chip, including several Artificial Intelligence (AI) accelerators.

The results and conclusions of this work come from the experiments performed on the De-RISC platform. However, SafeDE is not integrated into the official version of the De-RISC project and the integration was made only as a research exercise. Regarding the SELENE project, SafeDE is one of the safety IP cores integrated into the platform's final version to cover its safety needs.
\bigskip

\subsection{Hardware Implementation and Integration}
\textcolor{red}{Add images}

Since the system only needs to interface with SafeDE to configure it by modifying its internal registers, it demands low bandwidth. Hence, to avoid unnecessary complexity is connected to the system through an APB interface. As shown in Figure \textcolor{red}{add reference}, SafeDE is built in three independent hardware layers developed in VHDL. 

\begin{itemize}
    \item The first layer (the inner one) called \textit{staggering\_handler} instantiates SafeDE's core functionality. This layer is responsible for calculating the instructions executed by each core, computing the staggering and asserting the stall signal when needed. This layer is portable since it is entirely independent of the platform interface. 
    \item The second layer implements the APB protocol. Configuration and statistical registers are implemented in this layer. This layer allows the AHB masters (cores) to modify SafeDE internal registers. The values stored in the internal registers act as inputs for the \textit{staggering\_handler} layer.
    \item Finally, the outer layer is an AMBA APB wrapper. This wrapper converts the defined APB AMBA types from the GRLIB IP library to standard VHDL types. In this way, the first two layers are completely portable for any platform implementing APB interface.
\end{itemize}

As mentioned, SafeDE needs an internal signal from each core pipeline that provides information about how many instructions each core has executed. SafeDE also needs a signal capable of stalling the cores, stopping their progress whenever the staggering is smaller than the set threshold. 

To provide the information related to the executed instructions, we employed a 2-bit signal "inst\_cnt". Each of the bits of the signal provides information about each issue of the core. Whenever the issue commits one instruction, the corresponding bit is set to '1' during one cycle. Each core is modified to feed these signals to SafeDE as inputs. With these signals, SafeDE can compute the executed instructions by both cores from a given time instant. Although the De-RISC and the SELENE platforms implement dual-issue NOEL-V cores, SafeDE can be configured through a generic to work with cores with an arbitrary number of issues. 

Each signal SafeDE uses to stall each of the cores is ANDed with a signal, called holdn, driven in the caches controller. The holdn signal is an active-low signal that freezes the pipeline of the core when it is set to '0', preventing the pipeline registers values from updating. ANDing the holdn and the SafeDE stall signal, the pipeline is held any time one of those signals is set to '0'. For simplicity, SafeDE stall outputs are active-high and their value is inverted later. 

Although it is not among the functionalities for which the SafeDE concept was initially devised, SafeDE also can be configured to set a maximum staggering threshold TH\_stag\_max. Whenever the staggering between both cores is bigger or equal to the configured threshold (\#instr\_head - \#instr\_trail >= TH\_stag\_max), the head core is stalled. This is useful to reduce the error detection time when intermediate results of the execution are compared between both cores.

The figure \textcolor{red}{add reference} shows an image of the De-RISC platform with SafeDE controlling the staggering of both cores. As shown in the figure \textcolor{red}{add reference}, SafeDE has 3 32-bit configuration registers (Configuration, CritSec1 and CritSec2) plus several statistic registers to gather execution information.

The bit 31 of the configuration register is used to perform a reset. When this bit is set to '1', SafeDE registers are set to their initial value except for the configuration register. The bit 30 of the configuration register is anded to the output stall signals. Thus, SafeDE is totally neutral and incapable of performing any action over the pipeline if this bit is not set to '1'. The rest 30 bits are used to configure maximum and minimum thresholds for the staggering (15 bits each). Whenever the staggering is bigger or equal to the maximum threshold or smaller or equal than the minimum threshold, the head or trail core will be stalled to keep the staggering within the established limits.  

The first bit of the register CritSec1 must be set to '1' when the core 1 enters the code region needing lockstepped execution. From the moment this bit is set to '1', SafeDE will start counting the instruction executed by core 1. Analogously, core 2 will set the first bit of the register CritSec2. The count of executed instructions will be reset for both cores once the first bit of both CritSec registers is set to '0'. Instead of using two bits of the same register, two different registers (one for each core) are implemented to indicate SafeDE the beginning of the region needing protection. The reason is that two different registers prevent writes from different cores from overriding the other core's bit.

SafeDE statistic registers gather information from the execution. Namely, they collect the following information: Total number of cycles SafeDE has been active, number of executed instructions by each core, number of times each core has been stalled during the execution, number of cycles each core has been stalled during the execution and maximum, minimum and average staggering during the execution.


\bigskip

\subsubsection{Configuration and Operation}

When using SafeDE, the first step is configuring the desired staggering thresholds TH\_stag\_max and TH\_stag\_min. If the TH\_stag\_max is not configured, TH\_stag\_max will be internally set close to its maximum possible value (32750). After configuring the staggering thresholds, the next step is to set to '1' the activation bit (bit 32 of the configuration register), which is Ored with the stall signals. The first core reaching the code section that needs lockstep protection will set its CritSec register first bit to '1'. From that moment, this core will assume the head roll and SafeDE will start counting each committed instruction. Later, the second core will reach the code region needing protection and set its CritSec register assuming the trail roll. Once one core have set it CritSec register, SafeDE will check that the staggering is within the limits (TH\_stag\_min < staggering < TH\_stag\_max) and will stall any of the cores if needed until the previous condition is met.  

Once the head core has finished the execution, it will set its CritSec register bit to '0'. SafeDE will let the trail core finish the execution without intervention. Once the trail core finishes the execution, it will set its CritSec register bit to '0' analogously to the head core. When both CritSec registers' first bits have value '0', SafeDE will set the instruction counts of both cores to 0.
\bigskip

\subsubsection{Software Integration}
\label{section:software_integration}
\textcolor{red}{bare metal and Linux integration}
\textcolor{red}{Put something similar to what is written in the journal}
\bigskip


%%% RESULTS %%%
\clearpage\section{SafeDE Evaluation}

This section assesses SafeDE by different methods in the context of De-RISC MPSoC. To perform the evaluation process, we simulated the VHDL model of the MPSoC, including SafeDE IP. For that purpose, we used the simulator QuestaSim. We also synthesized the platform into a Xilinx Kintex UltraScale KCU105 evaluation kit and performed several tests. 

\subsection{Functional Validation}

This section assesses SafeDE by different methods in the context of De-RISC MPSoC. To perform the evaluation process, we simulated the VHDL model of the MPSoC, including SafeDE IP. For that purpose, we used the simulator QuestaSim. We also synthesized the platform into a Xilinx Kintex UltraScale KCU105 evaluation kit and performed several tests. 

\subsection{Functional Validation}

Several strategies are applied to validate the correct SafeDE functionality:

\textbf{Testbench:} The first approach to validating SafeDE is using a VHDL testbench that simulates the system's behavior by generating SafeDE inputs and reacting to its outputs. For that purpose, a component simulating the behavior of the cores is developed. This component generates the inst\_cnt signals for both cores, randomly setting its bits to '1' as if the cores were committing instructions. If the stall signal of one core is set to '1', this prevents the inst\_cnt signal bits of that core from being set to '1' as it would in a real core. 

In addition to this component,  procedures are defined at the testbench top to simulate APB read and write transactions. These procedures allow configuring the internal SafeDE registers during the simulation. As mentioned before, SafeDE has some statistic registers that can be read through the APB interface. During the testbench execution, the same statistics are computed at the top of the testbench so they can be compared with the results recovered from SafeDE. 

Therefore, the testbench execution starts configuring internal SafeDE registers. After that, the testbench runs for a configurable number of cycles simulating two cores executing instructions that hold when the stall signals rise. Finally, SafeDE is stopped and the statistics are retrieved. The testbench pass provided that the statistics read from SafeDE registers and the statistics computed at the top of the testbench coincide and the staggering keeps all the time between the limits (TH\_max\_stagg > stag > TH\_min\_stag)

The testbench completion is the first step for SafeDE design validation and it also eases the development process since the testbench is automatically run each time a new Git push is performed, informing of potential errors each time the design is modified.



\textbf{RISC-V ISA tests:} RISC-V ISA tests \cite{ISAtests} are a group of tests designed by the University of California to test the correct functioning of the RISC-V ISA instructions. The ISA tests are written in a single assembly language file and contain a self-checking code to test the result. Each ISA test tests one operation from the RISC-V ISA forcing corner cases. 

To load the binaries into the platform, control the execution flow and debug the application, we used GRMON. GRMON is a general debug monitor for the LEON (SPARC V7/V8) processor, NOEL-V (RISC-V) and for SOC designs based on the GRLIB IP library developed by CAES Gaisler. 
\textcolor{red}{add reference to GRMON} 

We modified the RISC-V ISA tests assembly code to configure and activate SafeDE prior to the test execution. We also performed some modifications to perform the execution in two different cores. The linker script is modified to load two ISA tests in two different address segments. GRMON is employed to activate both cores and determine the correct entry point. Once the test has finished, the results are stored in one register file register in each core. Those registers are read using GRMON to test that each test successfully passed. 

This process is automated via Makefiles (for the compilation), Bash scripts and GRMON TCL scripts. Therefore, a simple Linux command compiles the binaries, uploads the binaries to the FPGA, controls the execution flow and checks the results for all of the selected RISC-V ISA tests. The correct completion of these tests proves that neither the internal modifications performed to the integer pipeline of the cores nor the SafeDE action produces any system malfunctioning.

\textbf{TACLe Benchmarks:} TACLe Benchmarks suite \cite{falk2016taclebench} is a set of self-contained and open-source benchmarks of varying types and sizes. They are specifically designed for the evaluation of critical real-time embedded systems. Since they are self-contained, they have their inputs hardcoded together with the source code, making them perfect for experimenting in a bare metal setup. Their execution times range from thousand of cycles to a few millions of cycles, making them, in many cases, easy to simulate and debug to understand unexpected results.

We ported some TACLe Benchmarks to RISC-V ISA and added some C functions from the bare metal drive to configure SafeDE and gather execution statistics. As with the RISC-V ISA tests, TACLe Benchmarks also have a self-checking function that compares the obtained results (or a signature summarizing the final result) and the expected results. TACLe Benchmarks are executed over the FPGA loaded with the De-RISC platform bitstream. Execution is controlled by GRMON. The compilation, loading and result checking processes are automated using different scripts. 

We checked for all the executed TACLe Benchmarks that the results with SafeDE forcing 20 instructions of staggering (TH\_stag\_min = 20; TH\_stag\_max = 32750) coincide with the expected results. Also, results from the statistic registers are gathered for every benchmark. We checked that in every execution, the number of executed instructions by both cores coincide and that the minimum staggering reached during the execution is never below that of the expected one (stag >= TH\_min\_stag = 20). 

Since a 1-cycle latency exists between the moment the stall signal is asserted until the core pipeline stops making progress, the former staggering condition does not hold all the time during the execution of the benchmarks. Since cores are dual-issue, they can commit two instructions each cycle, being the worst scenario the one in which, with a previous 21 instructions staggering, the trail core commits two instructions while the head core commits none. When this happens, the staggering reaches 19 instructions, and trail core stall signal rises. Since that signal will take effect the next cycle, the trail core can commit another two instructions while the head cores commit none again, leaving the staggering at 17 instructions. The same happens for TH\_max\_stagg. The real minimum staggering that is kept is dependent on the architecture, namely on the number of issues of the core and the number of cycles that elapses from the moment the stall signal is asserted until it stalls the pipeline. 

\bigskip



\subsection{Fault Injection}
\textcolor{red}{Put something similar to what is written in the journal}
\bigskip



\subsection{Time Overhead}

To determine how much SafeDE afects the performance we have run the TACLe Benchmarks in three different scenarios:

1) Isolation: Only one core executes the benchamrk while the other remains idle.
2) Redundancy without diversity: Both cores execute the same benchmark without SafeDE enforcing any staggering. 
3) Diverse Redundancy: Both cores run the same benchmark and SafeDE is active enforcing a staggering of 20 instructions (TH\_stag\_min = 20)

As stated before, is convinient that the staggering is big enough to ensure that pipelines of both cores do not contain the same instruction in any of the pipeline stages at any point of the execution. Taking into account that NOEL-V cores are dual-issue with a 7 stages pipeline (pipelines can execute in parallel 14 instructions), 20 instructions of staggering (17 in practice) will suffice to avoid the two pipelines executing the same instruction. 


%processes are generated loading binaries twice (one for each
%core) in different memory segments. We execute each bench-
%mark 1,000 times and use the average cycle count for each one
%for our evaluation to discount the effect of small variations
%due to, for instance, delays caused by DRAM refreshes. In
%any case, absolute variations observed are always in the order
%of few tens of cycles.

%To discount the effect of effects such as DRAM refreshes
%and other minor performance variations, we run each bench-
%mark 1,000 times and report average cycle counts. In any case,
%variations observed are up to few tens of cycles across runs.
%As shown in Figure 6, performance degrades only by 0.3%
%on average (up to 1.3% for BITONIC) w.r.t. isolation runs,
%and 0.003%, so ≈0% (up to 0.6% for IIR) w.r.t. redundancy
%without diversity. Performance variations across runs, and even
%marginal performance gains with SafeDE relate to the initial
%state of the branch predictor, and to memory alignment of the
%binaries impacting the instruction cache behavior, since in the
%case of SafeDE we execute additional instructions to configure
%SafeDE. The relative effect of those variations can be larger
%for short programs, as it is the case for FAC, with variations
%in the range of 1%. In fact, those variations have a higher
%impact than the tiny performance degradation of SafeDE w.r.t.
%redundancy without diversity.
%Overall, performance overheads are tiny due the very low
%staggering threshold needed by SafeDE (20 cycles in our case),
%which is far lower than that for the software-only solution (e.g.
%100μs or more) [3].



%Results are shown in Figure 4. As shown, in all the cases,
%the execution time overhead with SafeDE w.r.t. the execution
%time in isolation and in two cores without SafeDE is negligible.
%In particular, SafeDE causes an execution time degradation in
%most of the cases below 0.5%, and up to 1.3% in one case
%(BITONIC benchmark) w.r.t. the execution time in isolation.
%If we compare it against the redundant execution without
%enforcing diversity, execution time degradation is generally
%below 0.1% (in some cases performance even marginally
%improves), and up to 0.6% for IIR benchmark.
%In some cases, minor performance variations between the
%isolation and redundant versions of the programs are observed,
%being those differences neither caused by interference between
%redundant threads, nor by SafeDE operation itself. Instead,
%those variations are caused by the initial core state (e.g. branch
%predictor state), or changes in instruction cache behavior due
%to changes in the memory alignment of the binaries with and
%without thread redundancy. In the case of FAC benchmark,
%since it is a small benchmark (around 700 instructions only),
%these tiny effects have a visible impact in relative terms
%(e.g. 1% execution time increase without SafeDE and 0.1%
%decrease with SafeDE).
%Overall, as expected, execution time increase can be re-
%garded as negligible since the staggering threshold can be
%set very low (20 cycles in our evaluation), thus far below
%the 100μs, which correspond to many thousands of cycles,
%imposed by the software-only solution [3].
\bigskip

\subsection{Hardware Costs}

We have employed the Vivado 2018.1 toolchain to synthesize and generate the bitstream targeting the Xilinx UlstraScale KCU105 Evaluation Kit featuring the Kintex XCKU040-2FFVA1156E FPGA. SafeDE implementation required 261 LUTs and 417 registers. Those numbers are really low compared with the resources required by each core (approximately 38,000 LUTs and 17,000 registers) or with the hardware resources required by the whole MPSoC (approximately 114,000 LUTs and 74,000 registers). Thus, SafeDE hardware costs are negligible. Namely, just 0.23\% of the LUTs and 0.56\% of the registers of the whole MPSoC are employed to implement SafeDE. Hardware overhead could be limited even more by removing the statistics registers which are only useful for debugging porpuses.

\textcolor{red}{add image of the pie maybe?}
\bigskip




%%%% BDGET %%%
%\clearpage
%
%\section{Budget}
%{\selectlanguage{english}
%\foreignlanguage{english}{Depending on the thesis scope this document should include:}}

%%% ENVIRONMENT %%%
%\clearpage

%%% CONCLUSION AND FUTURE %%%
\clearpage
\section{Conclusions and future development: }

{This should include your summary, conclusions and recommendations. }

%%% BIBLIOGRAPHY %%%
\newpage

\medskip
\bibliographystyle{unsrt}
\bibliography{bibliography.bib}

%%% ANNEX %%%
\clearpage
\newpage

\begin{appendices}

{Appendices may be included in your thesis but it is not a requirement.}

\end{appendices}

\end{document}
